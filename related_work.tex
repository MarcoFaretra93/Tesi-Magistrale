In questa sezione andremo ad analizzare i sistemi simili (competitor) al Vadalog Reasoner, analizzandoli e paragonandoli al nostro sistema. \newline
Il capitolo è composto da una sola sezione (5.1) che descrive i confronti fatti tra il Vadalog Reasoner e i suoi competitor e le analisi fatte su questi ultimi. \newline

\section{Analisi e confronti}

Sono presenti molti sistemi esistenti simili al Vadalog Reasoner. I principali sono Graal~\cite{baget2015graal}, Llunatic~\cite{geerts2014s}, PDQ~\cite{benedikt2015querying} e DLV$^E$~\cite{leone2012efficiently} che condividono molto ma non tutto il potere espressivo del nostro sistema. \newline
La differenza principale di tutti questi sistemi è che nessuno da un grosso peso alla scalabilità, e quindi alla gestione dei Big Data, al contrario il Vadalog Reasoner ne fa un vero e proprio punto di forza. Inoltre il Vadalog Reasoner è l'unico tra tutti questi sistemi ad adottare un linguaggio della famiglia Warded Datalog$^\pm$\newline \newline
il sistema Graal è un toolkit implementato in Java, considera basi di conoscenza composte da dati e ontologie espresse da regole esistenziali. È basato su Datalog$^+$, quindi supporta le regole esistenziali, ma non garantisce decidibilità e tracciabilità dei dati al contrario del Vadalog Reasoner. \newline \newline
Llunatic offre un linguaggio da poter utilizzare dall'utente finale ed un'interfaccia grafica (applicazione desktop), è focalizzato principalmente sul mapping e il cleaning dei dati, che lo rendono molto limitato rispetto al Vadalog Reasoner. \newline \newline
PDQ è un sistema che è focalizzato principalmente sul reasoning, ha dei wrapper che permettono soltanto l'integrazione di dati web e provenienti da DBMS relazionali, inoltre l'algoritmo di ottimizzazione e ricerca ha un costo computazionale molto alto, come possiamo anche vedere nei test di cui abbiamo parlato nella sezione 4.2.2, in Figura~\ref{fig:ibenchgrafico} è il sistema con le prestazioni peggiori. \newline \newline
Infine DLV$^E$ è il sistema che più si avvicina al Vadalog Reasoner, infatti anch'esso utilizza un linguaggio Datalog$^\pm$, dove attraverso delle semplificazioni fornisce la decidibilità P-completa. La differenza sostanziale sta nel fatto che il Vadalog Reasoner è esteso da molte altre funzionalità che ne permettono l'applicazione in diversi scenari. \newline \newline
Molti di questi sistemi hanno una buona implementazione, tuttavia non posseggono un supporto specifico per le funzionalità richieste in importanti scenari aziendali e tendono a concentrarsi sull'approccio logico, anziché sull'architettura.


