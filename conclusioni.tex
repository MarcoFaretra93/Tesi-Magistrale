L'obiettivo di questa tesi è stato l'implementazione di feature core del Vadalog Reasoner per l'esecuzione di programmi Vadalog. \newline \newline
Durante quest'attività di tesi ho acquisito molte competenze, che troviamo di seguito:
\begin{itemize}
	\item Sviluppo del core di un Reasoner.
	\item Gestione di un'architettura stream.
	\item Conoscenza avanzata del linguaggio Datalog e le sue estensioni.
	\item Ottimizzazioni delle query.
	\item Creazione di benchmark efficienti per testare determinate caratteristiche.
	\item Tecniche di gestione della cache.
\end{itemize}
Inizialmente era presente una versione base, che permetteva il reasoning senza applicarne ottimizzazioni, poche sorgenti, pochi tipi di dato, poche funzionalità aggiuntive ed erano stati effettuati pochi benchmark per testare le perfomance. \newline
Durante la mia attività di tesi ho implementato diverse feature che colmando i problemi presenti, rendendo il sistema più stabile e ottimale come descritto nei capitoli precedenti. \newline \newline
Possibili sviluppi futuri sono principalmente la creazione di applicazioni mirate a modelli di business da presentare ad aziende interessate al prodotto e l'ampliamento di funzionalità offerte dal Vadalog Reasoner come l'integrazione di ulteriori storage con cui interagire, l'implementazione di nuovi algoritmi di join per ottimizzare scenari differenti, supporto a nuovi tipi di dato strutturati e tanto altro.